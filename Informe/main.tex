\documentclass{article}
\usepackage[utf8]{inputenc}
\usepackage[spanish]{babel}
\usepackage{listings}
\usepackage{graphicx}
\usepackage{cite}

\begin{document}

\begin{titlepage}
    \begin{center}
        \vspace*{1cm}
            
        \Huge
        \textbf{Informe Parcial 1}
            
        \vspace{0.5cm}
        \LARGE
        Informática II
            
        \vspace{1.5cm}
            
        \textbf{Daniel Perez\\Miguel Serna\\Jorge Montaña}
            
        \vfill
            
        \vspace{0.8cm}
            
        \Large
        Despartamento de Ingeniería Electrónica y Telecomunicaciones\\
        Universidad de Antioquia\\
        Medellín\\
        Abril de 2021
            
    \end{center}
\end{titlepage}

\tableofcontents

\section{Análisis del problema}
El primer reto era evidente, debíamos conectar 64 luces LED a la mínima cantidad de pines posibles, claro esta, haciendo uso del dispositvo 74CH595 presentado en la clase, para si poder ampliar la cantidad de salidas digitales.\\

Terminado con eso, podríamos empezar a buscar metodos para encerder ciertos LED y dejar los otros apagados y asi poder formar figuras con la matriz 8x8, la opción mas lógica sería primero encender todos los LED y poner dicha solución en la función 'Verificación'. Apartir de la solución usada en la función 'Imagen', la usaríamos en 'Publik' para que almacene los 3 caracteres deseados e imprima uno por uno.\\

Además de eso, debíamos hacer uso del serial para darle indicaciones al usuario y para que pudiese ingresar el numero, letra o caracter deseado y  aclarar las restricciones en el manual de usuario para así no generar errores.

\section{Esquema} \label{contenido}
texto aqui
\section{Algoritmo Implementado} \label{contenido}
\begin{lstlisting}
char texto = "aqui"
\end{lstlisting}

\section{Problemas de desarrollo} \label{conclulsion}
El principal de los problemas que se nos presento fue sobre como ibamos a indicar los valores que los LED debian tener para formar cada letra, al final decidimos realizarlo de la forma mas obvia y lógica, simplemete dando los valores en binario a cada fila para formar dicho caracter, por ejemplo, una fila completamente encendida seria [1,1,1,1,1,1,1,1] y con esa misma logica podriamos  'dibujar' las matrices fila por fia de cada letra. \\
se nos genero un error de conexion de los LEDs porque se formaban las figuras, pero al reves

\section{Evolución del algoritmo} \label{conclulsion}
evolucion here
\bibliographystyle{IEEEtran}
\bibliography{references}

\end{document}